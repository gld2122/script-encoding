\section{Two Unicode Stories in Southeast Asia}

\subsection{Vietnam}

\textcite{my13} considers problems and solutions to digital representation of
Vietnamese minority languages, specifically Ede (Rade). See the following key passage:

\begin{aquote}{\textcite{my13}}
In Vietnam, processing the Vietnamese-Kinh language problems has deployed fairly
soon, had many results, and has been continued. \textbf{However, the script
problem on the computer for the ethnic minority languages has not been
interested much.} Especially, in the explosive development of information and
communication technologies as well as internet, the services on the internet
have been relatively familiar to the people in almost all regions of the
country. \textbf{However, there is not any website in ethnic minority
languages.} Even in the website of the Committee for Ethnic Minorities
Vietnamese CEMA, there is not any ethnic minority language, the websites of the
locals where the ethnic people live are only in Vietnamese-Kinh language, or
accompanied by English.
\end{aquote}

Digital resources for the Vietnamese state language developed relatively quickly
due to economic incentives and state resources. (Vietnam's Ministry of Science,
Technology, and the Environment adopted a Vietnamese 8-bit ASCII extension,
Vietnamese Standard Code for Information Interchange (VSCII), in 1993
\parencite{vscii93}.) By the time of the paper, Unicode had replaced legacy
standards and Vietnamese typing tools in Unicode had been developed
\parencite{nguyen18}. Support lagged behind, however, for Vietnam's minority
languages, like Ede; it was necessary to implement an elaborate conversion
scheme to represent Ede-specific characters. Ede is written using Vietnamese
Latin characters, but includes some letters not present in Vietnamese. (For
example, the letter {\unicode Ƀ} used in Ede was not supported in Unicode until 2006, while
Vietnamese typing tools like Unikey and Vietkey were developed prior to this
inclusion. \parencite{b-with-stroke}) \textcite{my13} essentially present an
\textit{ad hoc} encoding scheme for the language in their Microsoft Word plugin
that adapts the Unicode-based typing tools to handle Ede characters.

The crucial benefit of Unicode in Vietnam is portability and
intercompatibility. Unicode provides an extensible framework; more languages may
be easily incorporated as standards developed. The limitation of Vietnamese
typing tools is due to lack of upgrades to their software, rather than to lack
of ambition by the Unicode Consortium for supporting minority languages. As
\textcite{my13} state: "The script processing of the ethnic minority on the
computer has only been solved locally for each ethnic language, have not been a
national unity and have not satisfied the needs of the culture development and
integration of ethnic minority communities in Vietnam." Implementing Unicode
solutions for these languages will allow written regional languages to operate
seamlessly amongst each other and with the national language, Vietnamese.

\subsection{Myanmar}

In nearby Myanmar, a very different story played out. One early font and
encoding package for Burmese, Zawgyi, based on the earlier font Myazedi, gained
early market dominance in the country. Burmese script, like other Brahmic
abugidas (Devanagari, Tamil, etc.) consists of base letters with modifications
and diacritics inserted around the base to mark vowels and other features.
Although Unicode specifies that such features should be encoded separately and
intelligently combined by software into the correct output character, advanced
rendering took a long time to be widely supported. Zawgyi's predecessor Myazedi
instead encoded each combined form in its own code point, resulting in a
confusing situation:

\begin{aquote}{\parencite{hotchkiss16}}
In Zawgyi, there are six different ways to write the word "myo" that render a
superficially "correct" character, and many more if you allow for "incorrect"
variations that would look strange but still intelligible to a reader. A
computer, however, sees these variations as completely different words. Modern
Unicode, by contrast, has only one code point per element, and will only render
if the characters are encoded in the correct sequence, meaning that for each
word there is one and only one encoding.
\end{aquote}

Lack of adherence to Unicode standards had another unintended effect. The
nonstandard encoding pattern used by Myazedi/Zawgyi took up code space that had
been reserved for Myanmar's ethnic languages, like Shan, Mon, Kayah, and Karen
\parencite{hotchkiss16}. Thus Zawgyi excludes from the outset the inclusion of
Myanmar ethnic identities in the country's digital infrastructure.

For legacy reasons, Zawgyi remained the dominant encoding in the country at the
time of the article in 2016. By this time, in contrast, Unicode and UTF-8 had
become nearly ubiquitous throughout much of the world. Smartphones by Samsung
and Huawei ship with Zawgyi as default, while the popular web portal
planet.com.mm, includes instructions on how to install Zawgyi on your system.
Myanmar blogs also offered guides for setting up Zawgyi \parencite{hotchkiss16}.
It was seen as a homegrown, familiar font, in contrast to difficult and foreign
Unicode fonts.  Lack of Unicode adoption is driven by lack of information among
average users, thus limiting Myanmar's full participation in global digital
infrastructure and the representation of all linguistic traditions present
within its borders: "New users in Myanmar unknowingly become part of Zawgyi’s
existing user base, without knowing the hidden costs that will impede the future
of Myanmar’s digital society to be sustainable and inclusive."
\parencite{liao17}
