\section{Two Unicode Stories}

\subsection{Vietnam}

The chapter by \textcite{my13} discusses problems with representing Vietnamese
minority languages in digital form. See the following key passage:

\begin{aquote}{\textcite{my13}}
In Vietnam, processing the Vietnamese-Kinh language problems has deployed fairly
soon, had many results, and has been continued. However, the script problem on
the computer for the ethnic minority languages has not been interested much.
Especially, in the explosive development of information and communication
technologies as well as internet, the services on the internet have been
relatively familiar to the people in almost all regions of the country.
\textbf{However, there is not any website in ethnic minority languages.} Even in
the website of the Committee for Ethnic Minorities Vietnamese CEMA, there is not
any ethnic minority language, the websites of the locals where the ethnic people
live are only in Vietnamese-Kinh language, or accompanied by English.
\end{aquote}

Implementing digital resources in the Vietnamese state language proceeded
quickly due to adequate resources.  At the time of the paper, although Unicode
was an established standard for national languages like Vietnamese, as
Vietnamese typing tools in Unicode had been developed around the turn of the
century \parencite{nguyen18}. Support lagged behind, however, for Vietnam's minority languages like
Ede; thus it was necessary to implement an elaborate conversion scheme to
represent Ede-specific characters. Ede is written using Vietnamese Latin
characters, but uses a few letters not present in Vietnamese. In essence, the
authors created an \textit{ad hoc} encoding scheme for the language in their
Microsoft Word plugin.

For the authors, the crucial aspect of Unicode is portability and
intercompatibility. Unicode provides an extensible framework; more languages may
be easily incorporated as standards developed. So instead of the situation
described by the authors, in which "the script processing of the ethnic minority
on the computer has only been solved locally for each ethnic language, have not
been a national unity and have not satisfied the needs of the culture
development and integration of ethnic minority communities in Vietnam,"
implementing Unicode solutions for these languages will allow written regional
languages to operate seamlessly amongst each other and with the national
language, Vietnamese.

\subsection{Myanmar}

Nearby in Myanmar, a very different story occurred. One early font and encoding
package for Burmese, Zawgyi, gained early adoption in the country, based on the
earlier font Myazedi. Burmese script, like other Brahmic abugidas, contains
base letters with modifications or additions made around the base to signify
vowels and other features. Although Unicode standard specifies that such
features should be encoded separately and intelligently rendered into the
correct output character, this took a while to be widely supported. Myazedi
instead encoded each form separately resulting in a confusing situation:

\begin{aquote}{\textcite{hotchkiss16}}
In Zawgyi, there are six different ways to write the word "myo" that render a
superficially "correct" character, and many more if you allow for "incorrect"
variations that would look strange but still intelligible to a reader. A
computer, however, sees these variations as completely different words. Modern
Unicode, by contrast, has only one code point per element, and will only render
if the characters are encoded in the correct sequence, meaning that for each
word there is one and only one encoding.
\end{aquote}

The other significant problem is that the nonstandard encoding in Myazedi/Zawgyi
used code space that had been reserved for Myanmar's small languages, like
Shan, Mon, Kayah, and Karen.

For legacy reasons, however, Zawgyi remained the dominant encoding in the
country at the time of the article. Smartphones by Samsung and Huawei ship with
Zawgyi as default, and the popular web portal planet.com.mm, included
instructions on how to install Zawgyi on your system. Myanmar blogs also offered
guides for setting up Zawgyi \parencite{hotchkiss16}. It was seen as a
homegrown, familiar font, in contrast to difficult and foreign Unicode fonts.
Lack of Unicode adoption, driven by lack of information among common users, is
limiting Myanmar's participation in modern digital culture and its
representation of all the linguistic cultures present within its borders: "New
users in Myanmar unknowingly become part of Zawgyi’s existing user base, without
knowing the hidden costs that will impede the future of Myanmar’s digital
society to be sustainable and inclusive." \parencite{liao17}
