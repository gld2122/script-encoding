\section{Introduction}

Language is fundamentally a spoken phenomenon, but ever since the dawn of
writing, the specter of written form has pulled a psychological trick over
viewers. We view it as "the real thing" because we can hold it in our hands,
roll it around, and sniff the paper it's written on. We view it as "better",
since its author spent painstaking hours beneath the candlelight, honing each
sentence to discover how far they might drive the limits of their craft. I'm
looking at you, "Aliens For Dinner" \parencite{spinner94}. And let's not forget
about other classic hits like "The Bible", "The Qur'an", and "The Corpus Juris
Civilis"; the centrality of written texts to law and religion underpins the
prestige of writing.

But in many ways, writing is the province of the few. Before widely available
education in developed countries, literacy was a clear divider between haves and
have-nots. Although literacy is widespread in the developed world today, the use
of writing \textit{on a large scale} (ie. publication) remains out of reach for
many people. Powerful groups control traditional means of publication, such as
publishing houses and news outlets. Today the Internet is changing this picture
somewhat. Internet fora like blogs, social media sites, and websites are a
low-cost way of submitting work to the wider world. Although being discovered
outside your own network is difficult on Internet platforms, hosting sites at
least give individuals' work a venue and a chance of being accessed by others.

The Internet introduces equal access problems of its own. Along with levelling
access to media, the Internet is simultaneously levelling the language in which
we access it. Due to the early development of computers and networking in the
United States and to the economic power of English, English gained a tangible
head start as the language of the Web. This is rapidly changing in the case of
major languages, like Spanish, French, and Chinese as Unicode-based technologies
make representing a wide array of languages much easier than in the past.

The real linguistic losers in the Internet game are small minority languages. By
this term, we do not refer to isolated or nonstandard spoken languages or
dialects. We are interested in languages with a written tradition, but which are
relatively marginal to the state in which they are used. Examples might include
langauges like Galician, Sindhi, Assamese, and Tamazight. These languages have
not benefitted from state resources in developing digital linguistic
infrastructure to non-English languages and thus find themselves even further
behind.

First we will outline the goals and rationale of Unicode and UTF-8 for creating
a multilingual Internet. Then we will see how Unicode adoption at the state
level is the best means of promoting linguistic diversity on the web and
protecting small languages by looking at cases from Vietnam and Myanmar.
